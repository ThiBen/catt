\documentclass[a4paper]{article}
\usepackage[utf8]{inputenc}
\usepackage[english]{babel}
\usepackage[T1]{fontenc}
\usepackage{amsmath,amssymb,amsfonts}
\usepackage{mathpartir}
\usepackage{macros}

\title{Notes}

\begin{document}
\maketitle

\section{Les r\`egles pour les coh\'erences}
\[
  \inferrule
  {
    \Delta\vdash C:\TCat\\
    \Gamma\vdashps^C\\
    C:\TCat,\Gamma\vdash A : \Type \ C
   }
  {
    \Delta,\Gamma\vdash\coh:A
  }
\]
lorsque $\FV(A)=\set{C}\cup\FV(\Gamma)$

\[
  \inferrule
  {
    \Delta\vdash C:\TCat\\
    \Gamma\vdashps^C\\
    C:\TCat,\Gamma\vdash t\to^C u : \Type \ C \\
    C:\TCat,\partial^-(\Gamma)\vdash t \\
    C:\TCat,\partial^+(\Gamma)\vdash u
  }
  {
    \Delta,\Gamma\vdash\coh:t\to^C u
  }
\]
lorsque $\FV(t)=\set{C}\cup\FV(\partial^-(\Gamma))$ et $\FV(u)=\set{C}\cup\FV(\partial^+(\Gamma))$


\[
  \inferrule
  {
    \Delta\vdash C:\TCat\\
    (x:*C)\vdashps^C\\
    C:\TCat,(x:*C)\vdash A : \Type \ C \\
  }
  {
    \Delta,\Gamma\vdash\coh:A
  }
\]
lorsque $\FV(A)=\set{C}$


\section{Les r\`egles pour les foncteurs}
\[
  \inferrule
  {
    \Delta\vdash F:C\To D\\
    \Gamma\vdashps^C\\
    C:\TCat, D:\TCat, F:C\To D,\Gamma\vdash A : \Type \ D
   }
  {
    \Delta,\Gamma\vdash\coh:A
  }
\]
lorsque $\FV(t\to^Du)=\set{C,D,F}\cup\FV(\Gamma)$

\[
  \inferrule
  {
    \Delta\vdash F:C\To D\\
    \Gamma\vdashps^C\\
    \Gamma_F,\Gamma\vdash t\to^D u : \Type \ D \\
    \Gamma_F,\partial^-(\Gamma)\vdash t \\
    \Gamma_F,\partial^+(\Gamma)\vdash u
  }
  {
    \Delta,\Gamma\vdash\coh:t\to^D u
  }
\]
avec $\Gamma_F = C:\TCat, D:\TCat, F:C\To D$, et \\
lorsque $\FV(t)=\set{C,D,F}\cup\FV(\partial^-(\Gamma))$ et $\FV(u)=\set{C,D,F}\cup\FV(\partial^+(\Gamma))$

\[
  \inferrule
  {
    \Delta\vdash F:C\To D\\
    (x : *C)\vdashps^C\\
    \Gamma_F,(x : *C)\vdash A : \Type \ D \\
    \Gamma_{\target(F)}, (x : *C) \vdash A
  }
  {
    \Delta,\Gamma\vdash\coh:A
  }
\]
lorsque $\FV(A)=\FV(\target F)\cup\set{x}$

\section{Les r\`egles pour les transformations naturelles}

\[
  \inferrule
  {
    \Delta\vdash \tau:C\To D|F \to G\\
    \Gamma\vdashps^C\\
    \Gamma_\tau,\Gamma\vdash A : \Type \ D
   }
  {
    \Delta,\Gamma\vdash\coh:A
  }
\]
lorsque $\FV(t\to^Du)=\set{C,D,F,G,\tau}\cup\FV(\Gamma)$

\[
  \inferrule
  {
    \Delta\vdash \tau:C\To D|F \to G\\
    \Gamma\vdashps^C\\
    \Gamma_\tau,\Gamma\vdash t\to^D u : \Type \ D \\
    \Gamma_\tau,\partial^-(\Gamma)\vdash t \\
    \Gamma_\tau,\partial^+(\Gamma)\vdash u
  }
  {
    \Delta,\Gamma\vdash\coh:t\to^D u
  }
\]
lorsque $\FV(t)=\set{C,D,F,G,\tau}\cup\FV(\partial^-(\Gamma))$ et $\FV(u)=\set{C,D,F,G,\tau}\cup\FV(\partial^+(\Gamma))$

\[
  \inferrule
  {
    \Delta\vdash \tau:C\To D|F \to G\\
    (x : *C)\vdashps^C\\
    \Gamma_F,(x : *C)\vdash t\to^D u :  \Type \ D \\
    \Gamma_{\source (\tau)},(x : *C)\vdash t \\
    \Gamma_{\target(\tau)},(x : *C)\vdash u
  }
  {
    \Delta,\Gamma\vdash\coh:t\to^D u
  }
\]
lorsque $\FV(t)=\FV(\source(\tau)) \cup\set{x}$ et $\FV(u)=\FV(\target(\tau)) \cup\set{x}$

\section{Les règles générales}
\subsection{structure}
Pour tout entier n,

\[
  \inferrule
  { }
  {
   \vdash\TCat\ n : \TCat\ n+1
  }
\]

\[
  \inferrule
  {
    \Gamma\vdash C : \TCat\ n
  }
  {
    \Gamma\vdash \Hom\ C : \TCat\ n+1
  }
\]


\[
  \inferrule
  {  }
  {
    \vdash \Hom\ (\TCat\ n+1) \equiv \TCat\ n+1
  }
\]


\[
  \inferrule
  {
    \Gamma\vdash \Hom\ C
  }
  {
    \Gamma\vdash *_C: \Hom\ C
  }
\]


\[
  \inferrule
  { }
  {
    \vdash *_{\TCat\ n+1} \equiv \TCat\ n
  }
\]

\[
  \inferrule
  {
    \Gamma\vdash C : \TCat\ n\\     
    \Gamma\vdash u : *_C \\
    \Gamma\vdash v : *_C
  }
  {
    \Gamma\vdash *_C\ |\ u\to v: \Hom\ C
  }
\]
\[
  \inferrule
  {
    \Gamma\vdash t\ |\ a \to b : \Hom\ C\\
    \Gamma\vdash u : t\ |\ a \to b\\
    \Gamma\vdash v : t\ |\ a \to b
  }
  {
    \Gamma\vdash t\ |\ a \to b\ |\ u\to v: \Hom\ C
  }
\]


  
\subsection{cohérences}
\[
  \inferrule
{
    \Xi\vdash \Hom\ \mathfrak{C}\\
    \Delta\vdashps^\mathfrak{C} \\
    \Xi,\Delta\vdash C : *_\mathfrak{C}\\
    \Xi,\Delta\vdash D : *_\mathfrak{C}\\
    \Gamma\vdashps^C\\
    \Xi,\Delta,\Gamma\vdash A :\Hom \ D \\
  }
  {
    \Xi,\Delta,\Gamma\vdash\coh : A
  }
  \]
  Lorsque \FV(\Delta)\cup\FV(\Gamma) = \FV(A)

\[
  \inferrule
  {
    \Xi\vdash \Hom\ \mathfrak{C}\\
    \Delta\vdashps^\mathfrak{C} \\
    \Xi,\Delta\vdash C : *_\mathfrak{C}\\
    \Xi,\Delta\vdash D : *_\mathfrak{C}\\
    \Gamma\vdashps^C\\
    \Xi,\Delta,\Gamma\vdash t\ |\ u \to v :  \Hom\ D \\
    \Xi,\Delta,\partial^-(\Gamma)\vdash u\\
    \Xi,\Delta,\partial^+(\Gamma)\vdash v
  }
  {
    \Xi,\Delta,\Gamma\vdash\coh : t\ |\ u \to v
  }
  \]
  Lorsque \FV(\Delta)\cup\FV(\partial^-(\Gamma)) = \FV(u) \text{ et } \FV(\Delta)\cup\FV(\partial^+(\Gamma)) = \FV(v)


  \[
  \inferrule
  {
    \Delta\vdashps^\TCat \\
    \Delta\vdash C : *_\TCat\\
    \Delta\vdash D : *_\TCat\\
    \Gamma\vdashps^C\\
    \Delta,\Gamma\vdash t\ |\ u \to v :  \Hom\ D \\
    \partial^-(\Delta),\Gamma\vdash u\\
    \partial^-(\Delta),\Gamma\vdash v
  }
  {
    \Delta,\Gamma\vdash\coh : t\ |\ u \to v
  }
  \]
  Lorsque \FV(\partial^-(\Delta))\cup\FV(\Gamma) = \FV(u) \text{ et } \FV(\partial^+(\Delta))\cup\FV(\Gamma) = \FV(v)

\[
  \inferrule
  {
    \Delta\vdashps^\TCat \\
    \Delta\vdash C : *_\TCat\\
    \Delta\vdash D : *_\TCat\\
    \Delta\vdash *_D :  \Hom\ D \\
  }
  {
    \Delta,(x : *_C)\vdash\coh : *_D
  }
\]
  Lorsque \FV(\partial^-(\Delta)) = \FV(*_D) \text{ et } \FV(\partial^+(\Delta)) = \FV(*_D)
 

\section{Idées de choses à faire}
\begin{itemize}
\item automatisation (de l'associativité), possibilité de traduire les preuve
  de/vers Globular
\item formalisation de l'implémentation avec arguments implicites
\item comment faire des preuves dans le système (e.g. ajouter de la coinduction,
  etc.). Exemple 0 de preuve : montrer que $coh f$ (= cohérence unaire) est
  équivalent à $f$.
\item comment ajouter des $\Pi$ et $\Sigma$ au système de preuve
\item faire la preuve du lien avec la définition de Grothendieck-Maltsiniotis
\item définition des foncteurs
\item généraliser à la catégorie des catégories
\end{itemize}

\end{document}
